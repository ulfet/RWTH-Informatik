\documentclass[10pt]{article}
\usepackage[top=50pt,bottom=50pt,left=50pt,right=50pt]{geometry}
\usepackage{color}
\usepackage[utf8]{inputenc}
\usepackage[T1]{fontenc}
\usepackage{float}

% used for enumerate based on different symbols
% \begin{enumerate}[(a)] e.g.
\usepackage[shortlabels]{enumitem}

%	\usepackage{indentfirst}
\usepackage{amsmath}
\usepackage{graphicx}
\usepackage{hyperref} 

% beautifully indenting an environment
% args: left margin, right margin
% \begin{addmargin}[1em]{2em} 
% \end{addmargin}
\usepackage{scrextend}

% for centered p columns in tabular environment
% \begin{tabular}{|P{2cm}|P{40mm}|P{4cm}|}
\usepackage{array}
\newcolumntype{P}[1]{>{\centering\arraybackslash}p{#1}}
\newcolumntype{M}[1]{>{\centering\arraybackslash}m{#1}}

% set paragraph indent to 0
% if you want to indent some paragraph, use \indentfirst command in front of them
\setlength{\parindent}{0pt}
\newcommand{\forceindent}{\leavevmode{\parindent=1em\indent}}


% for scaling a tabular table
\usepackage{graphicx}
\usepackage{grffile}


% for comment block
\usepackage{verbatim}

% for debug purposes
% \usepackage{showframe}

% for setting section start value
% \setcounter{section}{6}

% for code writing
% \begin{lstlisting} code here \end{lstlisting}
\usepackage{listings}
\lstset{
	% numbers=left,
	breaklines=true,
	tabsize=2,
	literate={\ \ }{{\ }}1,
	%	language=C,
	basicstyle=\footnotesize\ttfamily, 
	stepnumber=1,
	%	aboveskip=-10pt,
}

% for dividing a page into many columns
\usepackage{paracol}


\newcommand{\indentitem}{\setlength\itemindent{25pt}}

\begin{document}


% so that newlines(enter key strokes) in plain text would be counted as authentic newlines
%\bgroup\obeylines

\author{
	Ulfet CETIN - 391819\\
	Alisafa GASIMOV - 392560\\
	Orkhan HUSEYNOV - 374297\\	
	Shreya KAR - 392325\\
	(names sorted on last name basis)	
}
\title{
	Functional Programming\\
	Assignment 1
}
%\date{April 2019}

%	\frontmatter
\maketitle

\textbf{Note:} respective solutions could be found in the archive file we send\\
(filenames: exercise1.hs, exercise2.hs, exercise3.hs, exercise4.hs)

\section*{Exercise 1 (Function types): ((0.5 + 0.5 + 1) + 1 = 3 points))}

\begin{itemize}

\item[a)] Give examples of Haskell function declarations with the following types and briefly explain their semantics. Your solutions must not ignore any of their arguments completely.\\
i) Bool -> Bool -> Int

	\begin{addmargin}[0em]{2em}
		\begin{lstlisting}
		-- counts how many of the the two inputs is true
		f1 :: Bool -> Bool -> Int
		f1 True True = 2
		f1 True False = 1
		f1 False True = 1
		f1 _ _ = 0
		\end{lstlisting}
	\end{addmargin}

ii) [Int] -> [Bool] -> Int
	\begin{addmargin}[0em]{2em}
		\begin{lstlisting}
		-- assumes that the first and the second argument is of same size
		-- check, in how many of elements of the first argument,
		--      the truth value of condition (greater than zero, in our case)
		--          is the same as the respective element of second argument
		f2 :: [Int] -> [Bool] -> Int
		f2 [] [] = 0
		f2 (x:xs) (y:ys)
		    | (x > 0) == y = 1 + (f2 xs ys)
		    | otherwise = 0 + (f2 xs ys)
		\end{lstlisting}
	\end{addmargin}

iii) [Bool] -> (Bool -> Int) -> [Int]

	\begin{addmargin}[0em]{2em}
		\begin{lstlisting}
		-- maps each bool value in first argument using the second argument
		-- second argument should be a function that takes as input a bool and produces an int
		--      an example for second argument could be a voltageMapper function given below
		-- example usage:
		--      f3 [True,False,True,True,False] voltageMapper
		-- answer:
		--      [5,0,5,5,0]
		f3 :: [Bool] -> (Bool -> Int) -> [Int]
		f3 [] mapFunc = []
		f3 (x:xs) mapFunc = (mapFunc x):(f3 xs mapFunc)
		
		-- maps voltages into the values that system accepts
		-- for computer systems, it is either 5V or 12V, we preferred 12V
		voltageMapper :: Bool -> Int
		voltageMapper True = 5
		voltageMapper False = 0
		\end{lstlisting}
	\end{addmargin}

\item[b)] Suppose that f has the type Bool -> [Int] -> Int.

What is the type of x y -> f ((f True x)>0) [y]?

	\begin{addmargin}[0em]{2em}
		\begin{lstlisting}
		-- \x y -> f ((f True x)>0) [y]?
		--                        (f True x) evaluates to Int
		--                     (       Int      >0) evaluates to Bool
		--                 f          Bool              [Int] evaluates to Int, as definition of f shows
		-- type of  " \x y -> f ((f True x)>0) [y] " is Int
		\end{lstlisting}
	\end{addmargin}


\end{itemize}


\section*{Exercise 2 (Lists): (2 + 3 = 5 points)
}
\begin{itemize}
	\item[a)] For each of the following equations, if possible, give pairwise different example values for x, y, and z such that the equation holds. Otherwise explain why such an assignment is not possible.
	
	i) [[x],[y]] == [y]:z
	\begin{addmargin}[0em]{2em}
		\begin{lstlisting}
		x = 1
		y = 1
		z = [[1]]
		\end{lstlisting}
	\end{addmargin}
	
	ii) ([x] ++ z):y == (x:z):y
	\begin{addmargin}[0em]{2em}
		\begin{lstlisting}
		x = []
		y = []
		z = []
		\end{lstlisting}
	\end{addmargin}
	
	iii) [ [ ] ] ++ ([x]:y) == ([x]:z)
	\begin{addmargin}[0em]{2em}
		\begin{lstlisting}
			If we simplify the left side of the giving equation, then we get the following:
			[[]] ++ ([x]: y) = [[]] ++ [[x], ...] == [[], [x], ...]
			
			If we look at the right-hand side of the giving equation and compare it with the
			simplified expression, then we can easily see that first elements of the lists are
			different. Because of that we can say that in any values of x, y and z such an 
			assignment is not possible.
			
			The simplified expression: [[], [x], ...] 
			The right-hand side of the given equation: ([x], ...) == [[x], ...]
			So, [[], [x], ...] can not be equal to [[x], ...].
		\end{lstlisting}
	\end{addmargin}
	
	iv) (x:y):z == (y ++ [x]):z
	\begin{addmargin}[0em]{2em}
		\begin{lstlisting}
		x = 5
		y = []
		z = []
		(x:y):z == (y ++ [x]):z
		\end{lstlisting}
	\end{addmargin}
	
	\item[b)] Consider the following patterns\\
	p1) ([x]++y):ys

	p2) (x:y)++ys

	and the following terms:

	t1) [[]]

	t2) [[1,2],[3]]

	For each pair of a pattern and a term, indicate whether the pattern matches the term. If so, provide the
appropriate matching substitution. Otherwise, explain why the pattern does not match the term.
Does there exist a term that is matched by p1 but not by p2? Justify your answer.
	
	\begin{paracol}{2}
		\begin{itemize}
			\item p1 - t1
			\begin{addmargin}[-5em]{-2em}
				\begin{lstlisting}
				Lets assign x=[] empty list
				then  inside of bracket we will get [[]] in very simple case.
				In second part of this expression, achieved result inside of bracket ([[]]) should be added to the ys element which should append to the new nested list,
				so in very simple case,
				        we can get [[[]],ys element] 
				        which can not be equalised with [[]]
				\end{lstlisting}
			\end{addmargin}
		\end{itemize}
	
		\switchcolumn
	
		\begin{itemize}
			\item p2 - t1
			\begin{addmargin}[-5em]{2em}
				\begin{lstlisting}
				p2) (x:y)++ys
				t1) [[]]
				
				x = []
				y = []
				ys = []
				
				does match
				\end{lstlisting}
			\end{addmargin}
		\end{itemize}
	
		
		\switchcolumn
		
		
		\begin{itemize}
			\item p1 - t2
			\begin{addmargin}[-5em]{2em}
				\begin{lstlisting}
				p1) ([x]++y):ys
				t2) [[1,2],[3]]
				
				x = 1
				y = [2]
				ys = [[3]]
				
				does match
				
				\end{lstlisting}
			\end{addmargin}
		\end{itemize}
		
		\switchcolumn
		
		
		\begin{itemize}
			\item p2 - t2
			\begin{addmargin}[-5em]{2em}
				\begin{lstlisting}
				p2) (x:y)++ys
				t2) [[1,2],[3]]
						
				x = [1,2]
				y = []
				ys = [[3]]
						
				does match
				\end{lstlisting}
			\end{addmargin}
		\end{itemize}
			
	\end{paracol}
	
\end{itemize} 


\section*{Exercise 3 (Programming): (2 + 2 + 3 + 3 = 10 points)}

Note that you may use constructors like [], :, True, False in all of the following subexercises. You may also
write auxiliary functions if needed or reuse functions from earlier subexercises (even if you did not manage to
implement them).

\begin{itemize}
	\item[a)] Write a Haskell–function myrem, where myrem x y is the remainder of the integer division when dividing
x by y. So for example, myrem 14 3 == 2. If y == 0 then myrem x 0 == x. If y < 0 then myrem x y
== myrem x (-y).
	myrem :: Int -> Int -> Int

	You may not use any predefined functions except comparisons, +, and -.

	
	\begin{addmargin}[0em]{2em}
		\begin{lstlisting}
		myrem :: Int -> Int -> Int
		myrem x 0 = x
		myrem x y
		    | y < 0 = myrem x (-y)
		    | x >= y = myrem (x-y) y
		    | otherwise = x
		\end{lstlisting}
	\end{addmargin}
	
	\item[b)] Write a Haskell-function count that given a list xs and an element x returns the number of occurences
of x in xs. E.g., count 2 [0,2,2,0,2,5,0,2] == 4 wheras count (-7) [0,2,2,0,2,5,0,2] == 0.\\
	count :: Int -> [Int ] -> Int

	You may not use any predefined functions except comparisons and +.
	
	\begin{addmargin}[0em]{2em}
		\begin{lstlisting}
		count :: Int -> [ Int ] -> Int
		count _ [] = 0
		count wanted (x:xs)
		    | wanted == x = 1 + count wanted xs
		    | otherwise = count wanted xs
		\end{lstlisting}
	\end{addmargin}

	
	\item[c)] Write a Haskell-function simplify that given a list xs returns a list of pairs as follows. The resulting list contains the pair (x,n) if and only if x occurs in xs n times and n > 0. E.g., simplify

	[0,2,2,0,2,5,0,2] == [(0,3),(2,4),(5,1)].

	simplify :: [ Int] -> [( Int ,Int )]

	You may not use any predefined functions except comparisons.
	
	\begin{addmargin}[0em]{2em}
		\begin{lstlisting}
		-- is the first argument the smallest element of the second argument?
		isTheSmallest :: Int -> [Int] -> Bool
		isTheSmallest _ [] = True
		isTheSmallest elem (x:xs)
			| elem <= x = isTheSmallest elem xs
			| otherwise = False
		
		-- my personal sort function to be used as helper
		mySort :: [Int] -> [Int]
		mySort [] = []
		mySort (x:xs)
			| isTheSmallest x xs = x:(mySort xs)
			| otherwise = mySort (xs ++ [x])
		
		-- helper function(s) for part c
		-- finds whether element is in the list
		isIn :: Int -> [Int] -> Bool
		isIn _ [] = False
		isIn elem (x:xs)
			| elem == x = True
			| otherwise = isIn elem xs
		
		-- find unique elements of a list
		-- second argument should start as empty list, i.e. []
		findUniquesHelper:: [Int] -> [Int] -> [Int]
		findUniquesHelper [] resultList = resultList
		findUniquesHelper (x:xs) resultList
			| isIn x resultList = findUniquesHelper xs resultList
			| otherwise = findUniquesHelper xs (resultList ++ [x])
		
		-- main findUniques function making use of the helper function above
		findUniques :: [Int] -> [Int]
		findUniques list = mySort ( findUniquesHelper list [] )
		
		-- counts how many times an element appears in a list
		countSingleElement :: Int -> [Int] -> Int
		countSingleElement _ [] = 0
		countSingleElement elem (x:xs)
			| elem == x = 1 + countSingleElement elem xs
			| otherwise = countSingleElement elem xs
		
		-- first argument should be the unique elements of the second list
		simplifyHelper :: [Int] -> [Int] -> [(Int, Int)]
		simplifyHelper [] _ = []
		simplifyHelper (unique:uList) targetList = 
			[(unique, countSingleElement unique targetList)] ++ simplifyHelper uList targetList
		
		-- counts which element appeared how many times in the list
		simplify :: [ Int ] -> [( Int , Int )]
		simplify [] = []
		simplify list = simplifyHelper (findUniques list) list
		\end{lstlisting}
	\end{addmargin}
	
	\item[d)] Write a Haskell-function multUnion that given two lists of pairs xs and ys concatenates these lists where
each “multiple occurence” is simplified as follows: If xs contains a pair (x,n) and ys contains (x,m),
then the result contains (x,n+m). You may assume that in both xs and ys an integer occurs at most once
as first entry of a pair. Moreover, assume that the lists are sorted in ascending order w.r.t. the first entry
of the pair. Make sure that the resulting list is sorted in ascending order w.r.t. the first entry of the pair
as well. E.g., multUnion[(0,3),(2,4),(5,1)] [(-1,1),(0,4)] == [(-1,1),(0,7),(2,4),(5,1)].

	multUnion :: [( Int ,Int )] -> [( Int ,Int )] -> [( Int ,Int )]\\
	You may not use any predefined functions except comparisons and +
	
	\begin{addmargin}[0em]{2em}
		\begin{lstlisting}
		-- get all unique elements from that two list
		findUniquesTupleVersionHelper :: [(Int, Int)] -> [Int] -> [Int]
		findUniquesTupleVersionHelper [] resultList = resultList
		findUniquesTupleVersionHelper ( (key,count):list ) resultList
		    | isIn key resultList = findUniquesTupleVersionHelper list resultList
		    | otherwise = findUniquesTupleVersionHelper list (key:resultList)
		
		-- takes two lists as an argument and finds the unique keys
		findUniquesTupleVersion :: [(Int, Int)] -> [(Int, Int)] -> [Int]
		findUniquesTupleVersion list1 list2 = mySort (findUniquesTupleVersionHelper (list1 ++ list2) [] )
		
		-- write a retriever from the lists
		countRetriever :: Int -> [(Int, Int)] -> Int
		countRetriever key [] = 0
		countRetriever key ((key1, count1):xs)
		    | key == key1 = count1
		    | otherwise = countRetriever key xs
		
		multUnionHelper :: [Int] -> [(Int, Int)] -> [(Int, Int)] -> [(Int, Int)]
		multUnionHelper [] _ _ = []
		multUnionHelper (u:uniqueList) list1 list2 =
		    let summed = ( countRetriever u list1 ) + ( countRetriever u list2 )
		    in ( (u,summed):(multUnionHelper uniqueList list1 list2) )
		
		multUnion :: [(Int, Int)] -> [(Int, Int)] -> [(Int, Int)]
		multUnion list1 list2 = multUnionHelper (findUniquesTupleVersion list1 list2) list1 list2
		\end{lstlisting}
	\end{addmargin}
\end{itemize}


\clearpage
\section*{Exercise 4 (Infix Operators): (2+1* points)}

Define a Haskell function $^\wedge$$^\wedge$$^\wedge$ in infix notation with the type declaration

\begin{addmargin}[1em]{2em} 
	($^\wedge$$^\wedge$$^\wedge$) :: [Int] -> [Int] -> Int
\end{addmargin}


such that the following holds for lists of equal length:
\begin{itemize}
	\item The function call xs $^\wedge$$^\wedge$$^\wedge$ ys evaluates to xs to the power of ys interpreted as vectors, where the negative
entries of ys are ignored. In other words, [x1, x2, ..., xn] $^\wedge$$^\wedge$$^\wedge$ [y1, y2, ..., yn] == x1 $^\wedge$ y1
* x2 $^\wedge$ y2 * ... * xn $^\wedge$ yn.
	
	\item For example [1, 4, 5] $^\wedge$$^\wedge$$^\wedge$ [7, 2, 3] evaluates to 1 $^\wedge$ 7 * 4 $^\wedge$ 2

	* 5 $^\wedge$ 3 == 2000 and [1, 4, 5] $^\wedge$$^\wedge$$^\wedge$ [5, -1, 0] evaluates to 1 $^\wedge$ 5 * 5 $^\wedge$ 0 == 1.
	
	\item xs $^\wedge$$^\wedge$$^\wedge$ ys * z, where xs and ys have type [Int] and z has type Int, is a valid expression.

	The function $^\wedge$$^\wedge$$^\wedge$ may behave arbitrarily if the two arguments have different lengths. You may not use any
predefined functions except *, $^\wedge$, and comparisons. You may, of course, use constructors like [] and :.

	You can get one bonus point if you solve the exercise even without using the predefined function $^\wedge$.
\end{itemize}

\bigskip

	\begin{addmargin}[0em]{2em}
		\begin{lstlisting}
		-- equivalent of ^ operator, user defined for bonus points
		myPowOp :: Int -> Int -> Int
		myPowOp _ 0 = 1
		myPowOp a b = a * (myPowOp a (b-1))
		
		(^^^) :: [Int] -> [Int] -> Int
		[] ^^^ _ = 1
		_ ^^^ [] = 1
		(x:xs) ^^^ (y:ys)
		    | y <0 = xs ^^^ ys
		    | otherwise = (myPowOp x y) * (xs ^^^ ys)
		
		infixl 9 ^^^
		\end{lstlisting}
	\end{addmargin}


\end{document}