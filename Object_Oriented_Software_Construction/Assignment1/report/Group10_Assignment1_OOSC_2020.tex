\documentclass[a4paper,12pt,oneside]{scrreprt}
\usepackage[latin1]{inputenc}
\usepackage[T1]{fontenc}
\usepackage{ae,aecompl}
\usepackage[english]{babel}
\usepackage{amsmath}
\usepackage{amssymb}
\usepackage{amsfonts}
\usepackage{amsthm}
\usepackage{graphicx}
\usepackage{wrapfig}
\usepackage{ulem}
\usepackage{cancel}
\usepackage{float}
\usepackage{color}
%\usepackage{titlesec}
\usepackage{geometry}
\geometry{verbose,a4paper,tmargin=25mm,bmargin=25mm,lmargin=15mm,rmargin=25mm}
%\titlelabel{\thetitle.\quad}
\usepackage{tabularx}
\usepackage{booktabs}
\usepackage{paralist}
\usepackage{textcomp}
\usepackage[official]{eurosym}

\renewcommand{\rmdefault}{phv}
\renewcommand{\sfdefault}{phv}

% our OWN imports for our use
\usepackage{listings}
\lstdefinestyle{ourJavaStyle}{
	language=Java,
	%numbers=left,
	%numbersep=8pt,
	stepnumber=1,
	tabsize=2,
	showspaces=false,
	showstringspaces=false,
	basicstyle=\ttfamily\scriptsize,
	keywordstyle=\color{blue}\ttfamily,
	stringstyle=\color{red}\ttfamily,
	commentstyle=\color{green}\ttfamily,
	breaklines=true
}

\newcommand*{\sourcepath}{../code/src/main/java/de/rwth/swc/group10}
\newcommand*{\testpath}{../code/src/test/java/de/rwth/swc/group10}

\renewcommand{\thesubsection}{\thesection.\alph{subsection}}
%\titleformat{\subsection}
%{\normalfont\fontfamily{phv}\fontsize{14}{17}\bfseries}{\thesubsection}{1em}{}

\begin{document}
	
\begin{tabular}{ccc}
	\begin{large} \textbf{Prof. Lichter} \end{large} &
	
	\begin{minipage}[H]{3.5cm}
	\centering
		\begin{large} OOSC \end{large} \\
		\begin{large} WS 2019/2020 \end{large}
	\end{minipage} &
	
	\begin{minipage}[H]{4cm}
		\includegraphics[keepaspectratio,width=\textwidth,angle=0]{images/swc.png}
	\end{minipage} \\
Andreas Steffens, Konrad F\"ogen &  &  \\
& \begin{huge} \textbf{Submission 1} \end{huge}&  \\
& oosc@swc.rwth-aachen.de &  \\
& & \\
 % Hier drunter muessen die Daten noch angepasst werden
Issued: 21.10.2019 &
Submission: 04.11.2019 &
Discussion: 07.11.2019 \\
\end{tabular}
\newline \newline \newline
\centering
Submitted by Group 10

\begin{tabular}{ll}
	% TODO: enter your matriculation number accordingly
	Dominik Bittner, & 369202 \\
	Ulfet Cetin, & 391819\\
 	Philipp Hochmann, & 356148 \\
 	Anar Orujov, & 391825\\
 	Ada Slupczynski, & 384147\\
 	(sorted on lastname basis)
\end{tabular}

\setcounter{chapter}{1} % Aktuelles Assigment
\section{Assignment: Reuse}

\paragraph{Assumptions:}
\begin{itemize}
	\item all will reuse the code with the same effort reduction
	\item all teams make the same amount of mistakes with respect to LOC
	\item the cost of the new code is per line of code
    \item Four systems to develop (\textbf{L}ife-, \textbf{A}utomobile , \textbf{D}ental and \textbf{I}ndemnity insurance).
    \item Costs for new code are 100\euro/LOC.
    \item Costs for an error are 15.000\euro.
    \item The existing system has 50KLOC.
    \item "`[..] when re-using components, you reduce your effort by 80\% as in
    the case where you would re-implement from scratch [...]"' $\rightarrow$ reuse a component for 20\% of the development costs $\rightarrow$ RCR=0,2
    \item Changes for reusing the old system are about 4KLOC.
    \item Modifying would require 70\% more effort $\rightarrow$ RCWR=1,7
    \item Reusing 30\% for the life-, 50\% for the dental, 20\% for the indemnity and 10\% for the automobile insurance system
\end{itemize}

\begin{table}[h]
	\centering
	\resizebox{\textwidth}{!}{%
		\begin{tabular}{|l|r|}
			\hline
			Relative Cost for Reuse (RCR) & 0,2      \\ \hline
			Relative Code of Writing for Resue (RCWR) & 1,7      \\ \hline
			ErrorRate       & 1,2/KLOC \\ \hline
			CostPerError    & 15.000\euro  \\ \hline
			CostForNewCode  & 100\euro/LOC \\ \hline
			NewCodeForReuse & 4 KLOC   \\ \hline
		\end{tabular}%
	}
\end{table}

\paragraph{Calculation:}
\begin{equation*}
	ROI = \sum_i(RCA_i) - ADC = (RCA_L + RCA_D + RCA_I + RCA_A) - ADC
\end{equation*}
\begin{equation*}
	DCA_i = \text{Percent of usage} * 50 \text{ KLOC} * (1 - RCR) * 100 \text{ \euro/LOC}
\end{equation*}
\begin{equation*}
	MCA_i = \text{Percent of usage} * 50 \text{ KLOC} * 1,2/\text{KLOC} * 15000\text{\euro}
\end{equation*}

Life-insurance:
\begin{equation*}
	DCA_L = 30\% * 50 \text{ KLOC} * 0,8 * 100\text{ \euro/LOC} = 1200000 \text{\euro}
\end{equation*}
\begin{equation*}
	MCA_L = 30\% * 50 \text{ KLOC} * 1,2/\text{KLOC} * 15000\text{\euro} = 270000\text{\euro}
\end{equation*}
\begin{equation*}
	RCA_L = DCA_L + MCA_L = 1200000\text{\euro} + 270000\text{\euro} = 1470000\text{\euro}
\end{equation*}

Dental insurance:
\begin{equation*}
	DCA_D = 50\% * 50 \text{ KLOC} * 0,8 * 100\text{ \euro/LOC} = 2000000 \text{\euro}
\end{equation*}
\begin{equation*}
	MCA_D = 50\% * 50 \text{ KLOC} * 1,2/\text{KLOC} * 15000\text{\euro} = 450000\text{\euro}
\end{equation*}
\begin{equation*}
	RCA_D = DCA_D + MCA_D = 2000000\text{\euro} + 450000\text{\euro} = 2450000\text{\euro}
\end{equation*}

Indemnity insurance:
\begin{equation*}
	DCA_I = 20\% * 50 \text{ KLOC} * 0,8 * 100\text{ \euro/LOC} = 800000 \text{\euro}
\end{equation*}
\begin{equation*}
	MCA_I = 20\% * 50 \text{ KLOC} * 1,2/\text{KLOC} * 15000\text{\euro} = 180000\text{\euro}
\end{equation*}
\begin{equation*}
	RCA_I = DCA_I + MCA_I = 800000\text{\euro} + 180000\text{\euro} = 980000\text{\euro}
\end{equation*}

Automobile insurance:
\begin{equation*}
	DCA_A = 10\% * 50 \text{ KLOC} * 0,8 * 100\text{ \euro/LOC} = 400000 \text{\euro}
\end{equation*}
\begin{equation*}
	MCA_A = 10\% * 50 \text{ KLOC} * 1,2/\text{KLOC} * 15000\text{\euro} = 90000\text{\euro}
\end{equation*}
\begin{equation*}
	RCA_A = DCA_A + MCA_A = 400000\text{\euro} + 90000\text{\euro} = 490000\text{\euro}
\end{equation*}

\begin{equation*}
	ADC =  (1,7 - 1) * 4\text{ KLOC} * 100 \text{\euro/LOC} = 280000\text{\euro}
\end{equation*}
\begin{equation*}
	ROI = (1470000\text{\euro} + 2450000\text{\euro} + 980000\text{\euro} + 490000\text{\euro}) - 280000\text{\euro} = 5110000\text{\euro}
\end{equation*}

The result is a ROI of 5.110.000\euro. So there will be a big benefit of reusing the existing system. Due to this result I would make a clear reuse decision.


\section{Assignment: Design by Contract}


\begin{enumerate}[a)]
	\item Assume you are given the following interface for a new enhanced Date
	implementation. Please provide an implementation with Design by Contract in Java 	using Assertions or Valid4J (http://www.valid4j.org/).\\
	
%	\vspace{2em}
% enable those commented out parts to have the code included in the pdf
	% \lstinputlisting[style=ourJavaStyle]{\sourcepath /DateInterface.java}
	% \lstinputlisting[style=ourJavaStyle]{\sourcepath /OOSCDate.java}
	
		\begin{itemize}
			\item One can find our implementation in our submission under "OOSCDate" class, as stated by the assignment text.
		\end{itemize}
	
	\item Make a robust implementation by using Defensive Programming.
	
		\begin{itemize}
			\item One can find our implementation in our submission under "OOSCDateDefensive" together with its interface "DateInterfaceDefensive".
			
			\item "DateInterfaceDefensive" is just "DateInterface" with throws declarations added.
		\end{itemize}
	
	\item  Write a set of JUnit tests to check the functionality and show how DbC and Defensive Programming works in your case.
	
		\begin{itemize}
			\item One can find the tests for DbC in our submission under "OOSCDateTest" class.
			
			\item One can find the tests for Defensive Programming in our submission under "OOSCDateDefensiveTest" class.
		\end{itemize}
	
	\item Explain the use of Design by Contract and Defensive Programming in your
	implementation. Are there commonalities or conflicts?
	
		\begin{compactitem}
			\item There are commonalities in the implementations of them.
				\begin{itemize}
					\item Both of our implementations nearly use the same code: however, just before every function, there is require() statements in the DbC approach, and there is "throw exception" statements in Defensive Programming approach.
					\item One difference is in DbC, at the end of the functions, we have ensure() statements, and in Defensive Programming approach, at the entrance of functions, we check for validity of inputs (and throw exceptions if needed), but we do not check again at the end of the said functions.
				\end{itemize}
		\end{compactitem}
\end{enumerate}

\section{Assignment: Design by Contract with Inheritance}
Based on the previous result, you are now required to extend your class to handle more details regarding Time (hours, minutes and seconds).

\begin{enumerate}[a)]
	\item Please extend your code with the necessary fields and methods. Also add new contract information and modify existing contracts if necessary.

		\begin{itemize}
			\item One can find our implementation in our submission under "OOSCDateTime" class together with its interface "DateTimeInterface".
		\end{itemize}

	\item Is it also necessary to extend the Interface?

		\begin{itemize}
            \item No, it is not necessary to extend the interface. The new methods could be added in the OOSCDateTime class and called from an object of this class directly.
            \item An extension inside the "DateInterface" is not possible because the "OOSCDate" class does not implement any of the hour/minute/second-methods.
			\item But we created a new interface "DateTimeInterface" which extends the "DateInterface". So it is possible to easily replace our OOSCDateTime class with an other implementation. 
            \item With the new interface the user of the class does not have to know about the real implementation. Instead he/she just uses an object of the interface type. In case of dependency injection the user doesn't even have to know the name of the OOSCDateTime class.
			\item One can find the extended interface in our submission under "DateTimeInterface".
		\end{itemize}

	\item Write a set of JUnit tests which checks the functionality and shows how DbC works in your case.

		\begin{itemize}
			\item One can find the tests in our submission under "OOSCDateTimeTest" class.
		\end{itemize}

\end{enumerate}
\end{document}
